\section{Conclusion}
Real-world traffic systems are complex, composed of light and heavy vehicles,
complex road systems, individual drivers etc. We have chosen to work
with a minimalistic model, still capable of reproducing phantom jams as
observed in real-world traffic.\\\\

We have investigated two systems that show promising results in our
simulations compared to a \emph{normal driver}. The normal driver used in our
simulations is, however, not a normal driver. It is capable of perfectly assessing
the distance to and the velocity of, the vehicle if front of it. All cars also
share the same driver model. In fact, the only thing separating the normal
driver system from the ACC system is the reaction time. We have done some
simulations with mixtures of vehicles with different dynamics and with
non-deterministic driver models. Our impression is that this worsens the
problem with phantom jams, and reduces traffic flow further. We also believe
that ACC and EACC has the ability to stabilize these systems, which more closely
resembles reality, and expect the performance gap to normal drivers to be even
larger in reality. But, more investigations and simulations on the topic is
need.\\\\

So, what's the catch? Using systems such as ACC and EACC in real world
situations to improve road capacity might not be a straight-forward task. G.
Marsden et al. address some of the problems with ACC in \emph{Towards an
understanding of adaptive cruise control} \cite{accCritics}; A lane with ACC
vehicles can experience increased instability, compared with manual driving
in some traffic situations. For instance, when a manually controlled vehicle
cut in between two ACC vehicles, a sharp deceleration caused by the suddenly
decreased time-gap might start a travelling traffic jam. Also, speed and
traffic capacity has been shown to vary with ACC target time-gaps and
penetration rates.\\\\

% F�rare st�nger idag av systemen i dense traffic - inte avsedda f�r detta
% idag? cite??

%dynamics vary with time-gap and implementeringsgrad % Nu ett stycke med
%nackdelar, sv�righeter och % fr�ga kring effekter av implementeringsgrad:
%Inte helt simpelt - adc kan �ka problem \cite!  Utformning av eadc, f�rslag
%och potential. V�r ide - men finns f�rslag \cite!  eacc beh�ver v�ldigt h�g
%implementeringsgrad eftersom de inte kan kommunicera annars Unders�k effekter
%av implementeringsgrad.

%% Sum -up conclusion
To sum up; Traffic jams constitute a severe problem in the world today.
Building new roads or modifying old road systems to reduce jams and improve
road performance costs huge amounts of money. Our simulations clearly
indicates that automatic or semi-automatic vehicle control systems have the
potential to shift the critical level of traffic density at which phantom jams
occur upwards. Also, the performance penalty for any occurring jams is
reduced. This means that, if such systems are successfully implemented,
existing roads could handle heavier or much heavier traffic.

\section*{Acknowledgements}
This work was performed as part of the course \emph{Simulation of Complex
Systems} at Chalmers University of Technology. A thanks to our advisor
Kolbj{\o}rn Tunstr{\o}m.

