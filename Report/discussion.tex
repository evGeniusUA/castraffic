\section{Discussion \and Conclusion}
Real-world traffic systems are complex, composed of light and heavy vehicles,
complex road systems, individual drivers etc. We have chosen to work
with a minimalistic model, still capable of reproducing phantom jams as
observed in real-world traffic.\\\\

We have investigated two systems that show promising results in our
simulations compared to a \emph{normal driver}. The normal driver used in our
simultions is, however, not a normal driver. It is capable of perfectly assesing
the distance to and the velocity of, the vehicle if front of it. All cars also
share the same driver model. In fact, the only thing separating the normal
driver system from the ACC system is the reaction time. We have done some
simulations with mixtures of vehicles with different dynamics and with
non-deterministic driver models. Our impression is that this worsens the
problem with phantom jams, and reduces traffic flow further. We also believe
that ACC and EACC has the ability to stabilize these systems, which more closely
resembles reality, and expect the performance gap to normal drivers to be even
larger. But, more investigations and simulations on the topic is need.

%% Nu ett stycke med nackdelar, sv�righeter och
%% fr�ga kring effekter av implementeringsgrad:
% Inte helt simpelt - adc kan �ka problem \cite!
% Utformning av eadc, f�rslag och potential. V�r ide - men finns f�rslag \cite!
% eacc beh�ver v�ldigt h�g implementeringsgrad eftersom de inte kan kommunicera annars
% Unders�k effekter av implementeringsgrad.

%% Sum -up conclusion
To sum up; Traffic jams constitute a severe problem in the world today. Building new
roads or modifying old road systems to reduce jams and improve road
performance costs huge amounts of money. Our simulations clearly indicates
that automatic or semi-automatic vehicle control systems have the potential
shift the critical level of traffic density at which phantom jams occur,
upwards. Also, the performance penalty for any occurring jams is reduced.
This means that existing road systems could handle heavier or much heavier
traffic.

