\section{Simulator setup}
In the simulator created there was a one-lane circular road with a length of
800 m. Different amount of cars could be placed on the track corresponding to
a certain traffic density. During one simulation this meant that the density
was constant since no cars could be added or removed during one run. Initially
all cars were positioned equally spaced on the circle, but then every car
was moved forward randomly between 0 and 1 metre to create some initial
perturbation that speed up the upcoming of phantom jams. The design of the
simulator can be seen in Appendix. (FIXME: picture of the simulator).

\subsection{Implementation of mathematical models}
The three systems described in Sections (FIXME: ref till dessa tre modeller)
were implemented as described below. Since the simulator used a circular road
position of the car was transformed into an angle from 0 to 2\begin{math}\pi
\end{math} but since the acceleration and velocity were not affected by this,
the car was only aware of a straight road where the car going out in one
end started over from the other end.

\subsubsection {Normal driver}
The IDM is developed to describe a normal behavior of cars in traffic
and hence we have implemented the model in our simulations with only one
difference. A delay of the acceleration has been added which represents a
reaction time. For human drivers it takes about 1 s to react to changes in
traffic \cite{idm}. Our model is then implemented as equation(\ref{driver_acc})
but with a time delay \begin{math}T_r\end{math} which affects the acceleration.

\subsubsection {Adaptive cruise control }
The purpose of the ACC is to keep a constant time gap to the car in front
and since IDM already has this ability only the reaction time of the model
has been changed between the normal driver and ACC-driver. The ACC system
is electronically controlled and we believe that the system has a reaction
time of about 200 ms. Table \ref{acc_config} shows the parameter setings.

\subsubsection {Enhanced adaptive cruise control}
How to implement a system that can adapt to changes further ahead than the
car infront is not obvious. In our model we have assumed that the system
can get the exact information about the position and velocity of the cars
further ahead.  The dynamics that should be considered from the car in front
is the difference of velocity between the two cars. This is implemented in
the effective desired distance equation (\ref{desireddist}) from
the IDM. The enhanced model was then realized by changing the equation of
the desired distance and also include the difference of velocity of the
car further ahead. To add this feature to the implemented model we changed
equation (\ref{desireddist}) to have one extra term.

\begin{equation}
s^\ast = s_0 + max(v_\alpha T + (1-\epsilon
)\frac{v_\alpha \Delta v}{2\sqrt{ab}} + \epsilon \frac{v_\alpha \Delta
v_2}{2\sqrt{ab}})
\end{equation}

where $ \Delta v_2 = v_\alpha - v_{\alpha +2} $.  Since the enhanced model
is controlled similar to the ACC system we also used the same parameters in
both systems. See table \ref{config} for parameter settings.

\begin{center}
\begin{table}[H]
\begin{tabular}{| l | l | l |} \hline
Paramter & Description & Value\\ \hline
a & Max acceleration & $ 0.73 \unit{m/s^2} $\\ \hline
b & Max brake & $ 1.5 \unit{m/s^2} $\\ \hline
T & Time headway & $ 1.5 \unit{s} $ \\ \hline
$ l $ & Car length & $ 5 \unit{m} $ \\ \hline
$ T_r $ & Reaction time & $ 1 \unit{s}, 0.2 \unit{s} $ (normal driver, ACC\/ EACC) \\ \hline
$ \epsilon $ & Communication influence & $ 20 \% $ (only for EACC) \\ \hline
\end{tabular}
\caption{\label{config} Parameters for the three models.}
\end{table}
\end{center}


