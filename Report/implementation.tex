\section{Simulator setup}
In the created simulator we had a one-lane circular road with a length of
800 m. Different amounts of cars could be placed on the track corresponding to
a certain traffic density. During one simulation this meant that the density
was constant since no cars could be added or removed during one run. Initially
all cars were positioned equally spaced on the circular road, but then every car
was moved forward randomly between 0 and 1 metre to create some initial
perturbation. This speeded up the emergence of phantom jams. The graphical user
interface of the simulator can be seen in Appendix \ref{gui}.

\subsection{Implementation of mathematical models}
The Intelligent Driver Model has been used as basis for the control system
for all agents (cars). We have done simulations with three different versions.

\begin{comment}
The three systems described in Sections (FIXME: ref till dessa tre modeller)
were implemented as described below. Since the simulator used a circular road
position of the car was transformed into an angle from 0 to 2\begin{math}\pi
\end{math} but since the acceleration and velocity were not affected by this,
the car was only aware of a straight road where the car going out in one
end started over from the other end.
\end{comment}

\subsubsection {Normal driver}
The IDM was developed to describe a normal behavior of cars in traffic
and hence for our normal driver, we have implemented the model with only one
modification. A delay of the acceleration has been added which represents a
human reaction time. For human drivers it takes about 1 s to react to changes in
traffic \cite{idm}. Our model is then implemented as equation (\ref{driver_acc})
but with a time delay \begin{math}T_r\end{math}.

\subsubsection{Adaptive Cruise Control}
The purpose of the ACC is to keep a constant time gap to the car in front
and since IDM already has this ability only the reaction time of the model
has been changed between the normal driver and ACC-driver. The ACC system
is electronically controlled and we assumed that the system has a reaction
time of about 200 ms. Table \ref{config} shows the parameter settings.

\subsubsection{Enhanced Adaptive Cruise Control}
It is not obvious how to implement a system that can adapt to changes further
ahead than the car in front. In our model we have assumed that the system
can get the information about the position and velocity of a car one step further
ahead. The data was used to the calculate the difference
in velocity of the two cars. This was then implemented through the effective desired
distance equation (\ref{desireddist}) from the IDM and resulted in the followin equation:
 
\begin{equation}
s^\ast = s_0 + max \left (v_\alpha T + (1-\epsilon
)\frac{v_\alpha \Delta v}{2\sqrt{ab}} + \epsilon \frac{v_\alpha \Delta
v_2}{2\sqrt{ab}}\right )
\end{equation}

where $ \Delta v_2 = v_\alpha - v_{\alpha +2} $. Otherwise the system is
similar to the ACC system and we have used the same parameters in both
systems. See table \ref{config} for parameter settings.

\begin{center}
\begin{table}[H]
\begin{tabular}{| l | l | l |} \hline
\emph{Paramter} & \emph{Description} & \emph{Value}\\ \hline
a & Max acceleration & $ 0.73 \unit{m/s^2} $\\ \hline
b & Max brake & $ 1.5 \unit{m/s^2} $\\ \hline
T & Time headway & $ 1.5 \unit{s} $ \\ \hline
$ l $ & Car length & $ 5 \unit{m} $ \\ \hline
$ T_r $ & Reaction time & $ 1 \unit{s}, 0.2 \unit{s} $ (normal driver, ACC\/ EACC) \\ \hline
$ \epsilon $ & Communication influence & $ 20 \% $ (only for EACC) \\ \hline
\end{tabular}
\caption{\label{config} Parameters for the three models.}
\end{table}
\end{center}


