\section{Methods to reduce jams}
A technology to increase safety for drivers in traffic is Adaptive Cruise
Control (ACC) which is the next generation of cruise control. This kind of
system is able to measure the distance to and speed of the car in front and then
adapt its own speed so a certain time gap is maintained between the cars. ACC
is already commercially available on the market and there is much research
going on to determine the effects on traffic flow when more and more vehicles are equipped with this system. The biggest advantage of the system is the increased comfort of the driver. A human driver is mostly not very good at estimating the distance to or the velocity of the car in front. This can cause unneccessary brakes or accelerations. Also because of some drivers behaviour, time headway between cars is shorter than a normal driver require to adapt to changes in traffic flow \cite{acc}.

% Nice intro EACC, snyggt sammanbundet!
One ability that human drivers have but ACC lack is the possibility to look
ahead in traffic. One example is the breaking light that can be seen through
several cars. A problem with this is the difficulty to estimate the speed of the
cars ahead. The only information available is that the cars further ahead are
breaking. We have thought of a system that have the advantages of the ACC and
the possibility to look further ahead in traffic. This enhanced model could
be realized by communication between the cars that are travelling in the same
direction. The enhanced system can then adapt speed to the cars further up in line and possibly reduce fluctuations in traffic flow even more. There has been some research on how this kind of intervehicle messaging would work  \cite{communication} but implementation details are not considered in our project.
