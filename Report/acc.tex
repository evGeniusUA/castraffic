\section{Methods to reduce jams}
A technology to increase safety for drivers in traffic is Adaptive Cruise
Control (ACC) which is the next generation of cruise control. This kind of
system is able to measure the distance and speed to the car infront and then
adapt the speed so a certain time gap is maintained between the cars. ACC
is already commercially available on the market and there is much research
going on to see the effect in traffic flow when more and more vehicles are
equipped with this system \cite{acc}. The biggest advantage of the system
is the increased comfort of the driver but also safety is increased. A human
driver is mostly not good at estimating the distance or the velocity to the car
in front which can cause unneccessary brakes or accelerations in the traffic
flow. Also because of some drivers behaviour the time headway between the
cars is even shorter than the reaction time of a normal driver which reduces
the ability of a driver to adapt to changes of traffic flow. Since ACC is
able to measure the distance and velocity with good precision and adapts
the speed to always keep a safe time headway to the car in front there isn't
neccessary to do as much braking and acceleration. (FIXME: ref till not)\\\\

One ability that human drivers have but not ACC is the possibility to look
ahead in traffic. One example is the braking light that can be seen through
several cars and this ability increses both safety and in some cases traffic
flow. A problem with this method is the difficulty to estimate the speed of the
cars ahead. The only information available is that the cars further ahead are
breaking. We have thought of system that have the advantages of the ACC and
the possibility to look further ahead in traffic. This enhanced model could
be realized by communication between the cars that are travelling in the same
direction. There has been some research on communication between cars and
(FIXME: Kesting et al. ) have tested the connectivity of such a system. The
enhanced system can then adapt the speed to the cars further up in line
depending of position and velocity and then possibly reduce stop \& go-traffic.
